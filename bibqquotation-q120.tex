[72]      This Court is alive to the issue on to what extent that our courts will recognize the legal standing of a particular entity. In 
%Edwards v. A.G. of Canada, 1929 CanLII 438 (UK JCPC), [1930] A.C. 124 (U.K.P.C.)
\lcinline{edwardsa2}, more commonly known as 
%“The Persons Case”
\lcnickname{edwardsa2}, the Privy Council overturned the Supreme Court of Canada and concluded that for the purposes of appointment of a senator 
%(The Constitution Act, 1867, 30 \& 31 Vict, c. 3, s. 24)
(\lawcite[24]{canconst1867b}), a ``person'' includes a female. More recently, Chief Justice Fraser of our Court of Appeal in 
%Reece v. Edmonton (City), 2011 ABCA 238, 335 D.L.R. (4th) 600, leave denied [2011] S.C.C.A. No. 447
\lccatena{%
reece;leave denied
,reece2;
} (in dissent) proposed that sentient animals should have legal standing, either directly or via advocates on their behalf: paras. 70, 88-91.
%\lcinline[1]{chow}
 

\par\bigskip
[73]      Nevertheless, in this instance I believe the correct approach is provided by an analogous scenario investigated in 
%Joly v. Pelletier, [1999] O.J. No. 1728 (QL), 1999 CarswellOnt 1587 (Ont. Sup. Ct. J.)
\lcinline{joly}. The plaintiff, Rene Joly, sued a variety of individuals, medical and lab facilities, and government officials who the plaintiff alleged had conspired with the American government to conceal the fact he was not human and: ``... to eliminate him and otherwise taken various steps to interfere with his ability to establish himself and live freely as a martian.'' Justice Epstein struck the action on two bases: it was frivolous and vexatious, and also as Rene Joly, self-admitted martian, did not have standing with the court:

 
\begin{quotation}
... While conspiracy to do harm to someone is the basis of many actions in the Court there is a fundamental flaw in the position of Mr. Joly. Rule 1.03 defines plaintiff as ``a person who commences an action''. The New Shorter Oxford English Dictionary defines person as ``an individual human being''. Section 29 of the Interpretation Act provides that a person includes a corporation. It follows that if the plaintiff is not a person in that he is neither a human being nor a corporation, he cannot be a plaintiff as contemplated by the Rules of Civil Procedure. \underLine{\highLight{The entire basis of Mr. Joly's actions is that he is a martian, not a human being.}} There is certainly no suggest[[ion]] that he is a corporation. \underLine{\highLight{I conclude therefore that Mr. Joly, on his pleadings as drafted, has no status before the Court}}. [Emphasis added.]
\end{quotation}

\par\bigskip 
[74]      Justice Epstein has adopted a strict approach to the definition of ``person''. In parallel, and absent clear legislative intent to the opposite, I refuse to entertain the directions of Ms. M. Leung (ghost), as channelled by Ms. Wong. This Court cannot and will not sit idly back and entertain applications by Ms. Wong that may be directed in Ms. Wong’s mind by the ghost of her late sister. To do so would be a clear abuse of process in the highest degree, and would bring the administration of justice into disrepute; something this Court will not permit. \endinput