

\begin{filecontents*}{\jobname.bib}
@xdata{cup,
  publisher    = {Cambridge University Press},
}

@collection{judges,
  editor       = {Paul Brand and Joshua Getzler},
  title         = {Judges and Judging in the History of the Common Law and Civil Law},
  subtitle     = {From Antiquity to Modern Times},
  date         = {2012},
  xdata        = {cup},
}

@incollection{precedent,
  author     = {Ian Williams},
  title        = {Early-modern judges and the practice of precedent},
  shorttitle = {Precedent},
  pages       = {51-66},
  crossref   = {judges},
}


@book{aandp,
author={Camillo Cavagnari and Emilio Caldara},
title = {Avvocati e procuratori},
editor = {Guido Alpa},
publisher = {il Mulino},
date = {2004},
}

\end{filecontents*}


%===================================================

\documentclass[12pt]{article}
\newcommand\rulesep{\rule{0.4\textwidth}{.4pt}}
\usepackage[table]{xcolor}
\pagecolor{blue!3}
\usepackage{fontspec}
\setmainfont{Noto Serif}
%\setsansfont{Noto Sans}[Scale=0.8]
%\setmonofont{Noto Sans Mono}%[Scale=0.8]

%------------------



\newcommand\abibname{lawcite}
\newcommand\abibstyle{style=\abibname}
\usepackage[
	\abibstyle , 
	indexing=cite,
	citetracker=true,
	ibidtracker=true,
	pagetracker=true,
	idemtracker=true,
	opcittracker=true,
	loccittracker=true,
	autocite=footnote,
   datezeros=true,
	lawcitestyle=default,
	use-toc-parnumrefs=true,
	comma-in-index=false,
		]{biblatex}

\addbibresource{\jobname.bib}
		
%=================
%indexing

\newcommand*\myhyperref[1]{\ref{para.#1}}

%\renewcommand\indexname{Index}

\usepackage[useindex]{splitidx}
\newcommand\pagerefindexnote{\bigskip\noindent\small\mdseries $\to$ References are to paragraph numbers.\bigskip}
\setindexpreamble[cases]{\pagerefindexnote}
\setindexpreamble[legislation]{\pagerefindexnote}
\setindexpreamble[regulations]{\pagerefindexnote}


%Print the `preamble' of the index:
\makeatletter
 \renewcommand*{\@@printindex}{}
 \def\@@printindex[#1][#2]{%
 \begingroup
 \edef\indexshortcut{#1}%
 \def\indexname{#2\par \useindexpreamble}%<=====
 \def\kvtcb@text@index{#2}%
 \let\index@preamble\relax
 \expandafter\let\expandafter\index@preamble
 \csname index@\indexshortcut @preamble\endcsname
 \if@splitidx
 \def\@tempa{idx}\def\@tempb{#1}%
 \ifx\@tempa\@tempb\let\@indexsuffix\@gobble\fi
 \fi
\@input@{\jobname\@indexsuffix{#1}.ind}%
\endgroup
\csname printindex@endhook\endcsname
}
\makeatother
\makeindex
\newindex[Index]{idx}
\newindex[Table of Cases]{cases}
\newindex[Table of Statutes]{legislation}
\newindex[Table of Regulations]{regulations}

%%%
%%%%Numbered paragraphs
%%%\newcounter{parno}%[paragraph]%% numbered paragraph
%%%\renewcommand{\theparno}{\arabic{parno}}
%%%\newcommand{\p}{\refstepcounter{parno}\noindent[\theparno]\label{para.\theparno}\ } 
%%%\setcounter{secnumdepth}{4}
%%%\setcounter{parno}{0}
%%%
%%%
%%%%numbered paragraphs cross-referencing method
%%%\newcommand\definname{para}
%%%\newcommand{\paradef}[1]{\label{#1}}
%%%\newcommand\pararef[1]{\definname\ [\ref{#1}]}
%%%



%-------------------------------------------------------
\AtWriteToIndex{cases}{\let\thepage\theparno}
\AtWriteToIndex{legislation}{\let\thepage\theparno}
\AtWriteToIndex{regulations}{\let\thepage\theparno}
%-------------------------------------------------------

\usepackage{tikz}
\usetikzlibrary{shadows,backgrounds}


%%%user bib commands
%%\newcommand\lcsec[1]{\bibstring{section}~#1}
%%\newcommand\lcsecyr[2]{\bibstring{section}~#1\addspace\mkbibparens{#2}}
%%\newcommand\lcssec[1]{\bibstring{sections}~#1}
%%\newcommand\lcssecyr[2]{\bibstring{sections}~#1\addspace\mkbibparens{#2}}

\usepackage[
				bookmarks,
            colorlinks=true,        
            allcolors = black,  
            citecolor=blue, 
            hyperindex=false,       
]{hyperref}

\newcommand\theball{\tikz \shade[ball color=green!40] (0,0) circle (0.75ex);\hspace{0.75em}}



%------------------
\begin{document}
\noindent Some normal citation commands.
\bigskip
%
%\hfill\rulesep\hfill\ %
%\bigskip
%
%\printlawciteindexes
%\bigskip
%\hfill\rulesep\hfill\ %\hrule%{0.8\linewidth}
%\bigskip
%
%
%\newpage

\bigskip

\setnumparshiftleft{\hspace{-2em}\theball\hspace{1.5em}}
\setnumparfillright{\hspace{0.5em}---\hspace{0.5em}}
%\setnumparformat{\color{blue}\bfseries}
%\setnumpardelimleft{}
%\setnumpardelimright{.}
\setnumpardelimpost{}

\p On the doctrine of \textit{stare decisis} and the binding nature of precedent: ``\ldots cases where judges are entirely constrained by previous cases aree relatively rare: given the opportunity, lawyers can (and lawyers did) distinguish cases not congenial to their argument rather than submit to an unwelcome earlier decision.'' \parencite[52]{precedent}.

\p The advocate defends, the proctor represents -- \cite[c 4, s 12; p 100]{aandp}. The book title, \citetitle{aandp}\footnote{The authors, \citeauthor{aandp}, are both advocates.}, could be translated under a Civil Law\index{Civil Law} lens as `Advocates and Proctors', which in Common Law\index{Common Law} terminology under a historical perspective would be `Barristers and Solicitors' in terms of original role;  or translated as `Defenders and Pleaders/Prosecutors', that is, `Defenders and Attorneys', in terms of their more modern roles.

\p To print a general index when \textsc{Biblatex}'s indexing is on (e.g., \texttt{indexing=cite}), use \texttt{splitidx}'s \texttt{useindex} option (\texttt{\textbackslash usepackage[useindex] \{splitidx\}}) and the corresponding \texttt{\textbackslash printindex[idx]} command. Custom items are sent to the general index with the standard \texttt{\textbackslash index\{\}} command. To get an index heading, \texttt{\textbackslash newindex[Index]\{idx\}} will be needed (and then \texttt{\textbackslash sindex[idx]\{\}} will also send an item to the general index). The toolchain continues to be xelatex--splitindex--xelatex.

\printbibliography

\printindex[idx]

\bigskip
\bigskip
\hfill ----oooOooo---- \hfill\ 
\end{document}